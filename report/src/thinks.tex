\section{Выводы}

В ходе выполнения лабораторных работ была построена полноценная базовая поисковая система на специализированном футбольном корпусе: от автоматизированной обкачки и очистки веб-страниц до токенизации, анализа распределения частот (закон Ципфа) и построения инвертированного булева индекса. Сформированный набор данных объёмом 34\,583 документа сочетает энциклопедический и публицистический стили, что делает его репрезентативным для задач информационного поиска; выбранная детерминированная токенизация обеспечивает высокую скорость и воспроизводимость, а наблюдаемое соответствие закону Ципфа подтверждает типичную статистическую структуру текста и пригодность корпуса для экспериментов с индексированием и ранжированием.

Анализ работы булевого поиска показал, что реализованная система корректно и быстро обрабатывает запросы различной сложности. Простые запросы из одного терма выполняются практически мгновенно, а более сложные выражения с операциями \texttt{OR} и \texttt{NOT} требуют большего времени из-за обработки крупных промежуточных списков документов, что является ожидаемым поведением для булева поиска. Полученные экспериментальные результаты подтверждают, что основная вычислительная нагрузка приходится именно на операции над posting list’ами, а не на разбор запросов или ввод-вывод. В целом все этапы работы — от построения корпуса до выполнения поисковых запросов — согласуются между собой и демонстрируют, что разработанная система является работоспособной, масштабируемой для корпуса данного размера и пригодной в качестве основы для дальнейших улучшений и усложнения поисковых моделей.


\pagebreak

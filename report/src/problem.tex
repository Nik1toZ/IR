\CWHeader{Задание}

Целью лабораторных работ является разработка базовых компонентов поисковой системы для работы с корпусом текстовых документов большого объёма.

В рамках выполнения лабораторных работ на оценку необходимо реализовать следующие компоненты:

\begin{enumerate}
    \item \textbf{Добыча корпуса документов.}  
    Сформировать и проанализировать уникальный корпус текстовых документов единой тематики объёмом не менее 30\,000 документов. Корпус должен состоять из текстов на естественном языке в кодировке UTF-8 и быть пригодным для дальнейшей обработки и индексирования.

    \item \textbf{Поисковый робот.}  
    Реализовать программный компонент для автоматической загрузки документов из сети Интернет или другого источника и сохранения их в локальном хранилище для последующей обработки. Робот должен обеспечивать корректную обработку ссылок и устойчивую работу при обкачке большого количества документов.

    \item \textbf{Токенизация.}  
    Реализовать разбиение текстов документов на токены (термы), пригодные для построения поискового индекса.

    \item \textbf{Стемминг.}  
    Реализовать алгоритм стемминга для приведения словоформ к общей основе. Алгоритм должен применяться ко всем токенам корпуса перед этапом индексирования.

    \item \textbf{Закон Ципфа.}  
    Провести статистический анализ корпуса документов и продемонстрировать выполнение закона Ципфа для распределения частот термов. Результаты анализа должны быть представлены в виде таблиц и графиков.

    \item \textbf{Булев индекс.}  
    Реализовать булевый индекс, позволяющий хранить информацию о вхождении термов в документы корпуса без использования готовых библиотек поисковых индексов.

    \item \textbf{Булев поиск.}  
    Реализовать обработку поисковых запросов с использованием булевых операций \texttt{AND}, \texttt{OR}, \texttt{NOT} над построенным индексом и обеспечить выдачу списка релевантных документов.
\end{enumerate}

Язык программирования для основных компонент — C++ . Для вспомогательных задач - Python.

\pagebreak

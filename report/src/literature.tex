\begin{thebibliography}{99}
\bibitem{Kormen}
Маннинг, Рагхаван, Шютце. \textit{Введение в информационный поиск} --- Издательский дом \enquote{Вильямс}, 2011. Перевод с английского: доктор физ.-мат. наук Д.\,А.\, Клюшина --- 528 с. (ISBN 978-5-8459-1623-4 (рус.)).


\bibitem{SAPE_robot}
Поисковый робот — что это: определение, работа и основные функции // Sape.ru. — URL: https://www.sape.ru/glossary/poiskoviy-robot/ (дата обращения: 13.11.2025).

\bibitem{HuggingFace_tokenizers}
Токенизаторы // HuggingFace LLM Course. — URL: {https://huggingface.co/learn/llm-course/ru/chapter2/4 }(дата обращения: 24.11.2025). 

\bibitem{Habr_search}
Основы полнотекстового поиска в ElasticSearch. Часть вторая // Habr.com. — URL: {https://habr.com/ru/companies/sportmaster_lab/articles/756270/} (дата обращения: 28.11.2025). 

\bibitem{Wiki_Zipf}
Закон Ципфа // Википедия. — URL: {https://ru.wikipedia.org/wiki/Закон_Ципфа} (дата обращения: 01.12.2025).

\bibitem{GeeksforGeeks_boolean}
Boolean Search: Meaning, Importance and Boolean Operators // GeeksforGeeks.org. — URL: {https://www.geeksforgeeks.org/hr/boolean-search-meaning-importance-and-boolean-operators/} (дата обращения: 12.12.2025). 

\end{thebibliography}
\pagebreak


